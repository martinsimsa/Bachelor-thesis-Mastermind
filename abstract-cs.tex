%%% Šablona pro samostatný abstrakt práce v češtině

% Meta-data o práci (je nutno upravit)
\input metadata.tex

% Vygenerujeme metadata ve formátu XMP pro použití balíčkem pdfx
\let\OrigThesisTitleXMP=\ThesisTitleXMP
\def\ThesisTitleXMP{\OrigThesisTitleXMP\space (abstrakt)}
\def\AbstractXMP{}
\def\ThesisKeywordsXMP{}
\input xmp.tex

\documentclass[12pt]{report}

\usepackage[a4paper, hmargin=1in, vmargin=1in]{geometry}
\usepackage[a-2u]{pdfx}
\usepackage[czech]{babel}
\usepackage{iftex}
\ifpdftex
\usepackage[utf8]{inputenc}
\usepackage[T1]{fontenc}
\usepackage{textcomp}
\fi
\usepackage{lmodern}
\usepackage{amsmath}
\usepackage{amsthm}
\usepackage{amsfonts}
\usepackage{fancyvrb}

\pagenumbering{gobble}

% Definice různých užitečných maker (viz popis uvnitř souboru)
\input macros.tex
\begin{document}
% Neřeknete-li jinak, abstrakt doplníme podle metadat
\Abstract
% Práce se zabývá popisem algoritmů řešících hru Mastermind pro libovolný počet pozic a barev. Stav hry je reprezentován dvěma způsoby, jako posloupnost pokusů s ohodnocením a jako množina zbývajících možností na tajný kód. Vazby mezi stavy jsou v obou případech popsány v grafu a mezi oběma grafy je nalezena spojitost. Pomocí těchto grafů je definován strom, který reprezentuje nějaký algoritmus pro hru Mastermind. Dále je sestrojen obecný algoritmus sloužící k~popisu konkrétních algoritmů. Ve třetí kapitole jsou popsány tři deterministické algoritmy, které hru řeší. Jde o algoritmy Min-max, Max entropy a Most parts. Nakonec jsou tyto metody implementovány a otestovány na hře Mastermind o~čtyřech pozicích a šesti barvách.
\end{document}