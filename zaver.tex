\chapter*{Závěr}
\addcontentsline{toc}{chapter}{Závěr}

Můj příspěvek v bakalářské práci spočíval ve vybudování definic 


Dokázal jsem tvrzení \ref{tvrzohodnoceni} a \ref{tvrzohodnoceni2} o existenci ohodnocení v prostorech kódů. Definoval jsem strom [n,k]-Mastermindu, multigraf prostoru kódů a [n,k]-Mastermindu a rozvinutí orientovaného acyklického multigrafu. Definoval jsem také strom algoritmu a dokázal tvrzení \ref{tvrzvlastnostistromualgoritmu} o jeho vlastnostech. Můj hlavní příspěvek spočíval ve zformulování obecného algoritmu pomocí funkcí valuace a strategie. Pomocí něj jsem odvodil tři již existující algoritmy Min-max, Max entropy a Most parts a dokázal jejich správnost. Ukázal jsem také chod Min-max algoritmu na [2,2]-Mastermindu. V části s entropií jsem vytvořil vlastní příklady, které vysvětlují myšlenku entropie. Také jsem podrobněji rozepsal důkaz věty o ekvivalenci maximalizace entropie \ref{vetaekvivalencemaxentropy}. Nakonec jsem tyto tři algoritmy implementoval v Pythonu a otestoval je na [4,6]-Mastermindu. 

V zadání bakalářské práce byl uveden článek od autorů Berghman, Goossens, Leus \cite{BERGHMAN20091880}, popisující genetický algoritmus řešící hru Mastermind. Tento algoritmus se ale od algoritmů popisovaných v této práci výrazně liší, a proto do práce nebyl zařazen. 

