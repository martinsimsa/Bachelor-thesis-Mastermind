\chapter*{Závěr}
\addcontentsline{toc}{chapter}{Závěr}

V této práci jsme vybudovali způsob znázornění možných stavů hry [n,k]-Mastermind jako strom [n,k]-Mastermindu. Také jsme vytvořili multigraf [n,k]-Mastermindu, který slouží k reprezentaci množin kandidátů stavu daného cestou z počátečního stavu. Mezi těmito dvěma grafy jsme našli přímou spojitost plynoucí z ekvivalentní konstrukce hran. Dále jsme popsali strom algoritmu, který znázorňuje průběh hry pro nějakou pevně zvolenou metodu. Sestrojili jsme také obecný algoritmus sloužící pro popis deterministických algoritmů. Dále jsme krátce popsali tři konkrétní algoritmy. Popsali jsme myšlenku entropie, která nemusí být úplně zřejmá a aplikovali ji v algoritmu Max entropy. Nakonec jsme algoritmy naprogramoval v Pythonu a našel jejich výsledky pro základní variantu hry. 

Dokázal jsem tvrzení \ref{tvrzohodnoceni} a \ref{tvrzohodnoceni2} o existenci ohodnocení v prostorech kódů. Definoval jsem strom [n,k]-Mastermindu, multigraf prostoru kódů a [n,k]-Mastermindu a rozvinutí orientovaného acyklického multigrafu. Definoval jsem také strom algoritmu a dokázal tvrzení \ref{tvrzvlastnostistromualgoritmu} o jeho vlastnostech. Můj hlavní příspěvek spočíval ve zformulování obecného algoritmu pomocí funkcí valuace a strategie. Pomocí něj jsem odvodil tři již existující algoritmy Min-max, Max entropy a Most parts a dokázal jejich správnost. Ukázal jsem také chod Min-max algoritmu na [2,2]-Mastermindu. V části s entropií jsem vytvořil vlastní příklady, které vysvětlují myšlenku entropie. Také jsem podrobněji rozepsal důkaz věty o ekvivalenci maximalizace entropie \ref{vetaekvivalencemaxentropy}. Nakonec jsem tyto tři algoritmy implementoval v Pythonu a otestoval je na [4,6]-Mastermindu. 

V zadání bakalářské práce byl uveden článek od autorů Berghman, Goossens, Leus \cite{BERGHMAN20091880}, popisující genetický algoritmus řešící hru Mastermind. Tento algoritmus se ale od algoritmů popisovaných v této práci výrazně liší, a proto do práce nebyl zařazen. 

