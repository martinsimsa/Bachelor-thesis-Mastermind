\chapter*{Závěr}
\addcontentsline{toc}{chapter}{Závěr}

V této práci jsme vybudovali způsob znázornění možných stavů hry [n,k]-Mastermind jako strom [n,k]-Mastermindu. Také jsme vytvořili multigraf [n,k]-Mastermindu, který slouží k reprezentaci množin kandidátů stavu daného cestou z počátečního stavu. Mezi těmito dvěma grafy jsme našli přímou spojitost pomocí rozvinutí. Dále jsme popsali strom algoritmu, který znázorňuje průběh hry pro nějakou pevně zvolenou metodu. Sestrojili jsme také obecný algoritmus sloužící pro popis deterministických algoritmů. Potom jsme krátce popsali tři konkrétní algoritmy. Popsali jsme myšlenku entropie, která nemusí být úplně zřejmá a aplikovali ji v algoritmu Max entropy. Nakonec byly algoritmy naprogramovány v Pythonu a nalezeny jejich výsledky pro základní variantu hry. 

Většina práce je mým vlastním přínosem. Části, kde jsem se inspiroval jinými zdroji jsou citovány přímo v textu. Jako hlavním přínosem teoretické části v kapitole $2$ považuji vytvoření multigrafu [n,k]-Mastermindu a propojení multigrafu se stromem algoritmu pomocí rozvinutí. Tento postup lze replikovat u jiných her založených na hádání tajného prvku. Dalším příspěvkem je také formální definice deterministických algoritmů pomocí valuace a strategie. Pomocí této terminologie jsem popsal konkrétní algoritmy a dokázal jejich správnost. Většina výsledků ve čtvrté kapitole již byla nalezena ve starších článcích. Tyto výsledky jsem úspěšně replikoval. Přidal jsem také porovnání algoritmů pro různé počáteční tahy. 


% Dokázal jsem tvrzení \ref{tvrzohodnoceni} a \ref{tvrzohodnoceni2} o existenci ohodnocení v prostorech kódů. Definoval jsem strom [n,k]-Mastermindu, multigraf prostoru kódů a [n,k]-Mastermindu a rozvinutí orientovaného acyklického multigrafu. Definoval jsem také strom algoritmu a dokázal tvrzení \ref{tvrzvlastnostistromualgoritmu} o jeho vlastnostech. Můj hlavní příspěvek spočíval ve zformulování obecného algoritmu pomocí funkcí valuace a strategie. Pomocí něj jsem odvodil tři již existující algoritmy Min-max, Max entropy a Most parts a dokázal jejich správnost. Ukázal jsem také chod Min-max algoritmu na [2,2]-Mastermindu. V části s entropií jsem vytvořil vlastní příklady, které vysvětlují myšlenku entropie. Také jsem podrobněji rozepsal důkaz věty o ekvivalenci maximalizace entropie \ref{vetaekvivalencemaxentropy}. Nakonec jsem tyto tři algoritmy implementoval v Pythonu a otestoval je na [4,6]-Mastermindu. 

% Většina práce je mým vlastním přínosem. V několika částech jsem se ale inspiroval jinými zdroji. Definice ohodnocení byla inspirována Knuthem \cite{donald_e__knuth_1977}. V definici podgrafu jsem se inspiroval Matouškem a Nešetřilem \cite{matouvsek2009kapitoly}. Termín množiny kandidátů byl inspirován různými publikacemi o Mastermindu. Termín orientovaně acyklický graf byl inspirovaný na webu \cite{DAG}. Algoritmy Min-max, Max entropy a Most parts byly převzaté z citovaných zdrojů. Způsob implementace pomocí valuace a strategie je ale mým přínosem. Myšlenka minimalizace $G$ v rovnosti \ref{rceocekavanypocetpokusu} byla také převzatá od Neuwirtha \cite{neuwirth}. Důkaz ekvivalence s maximalizací entropie jsem ale vytvořil samostatně. Příklady na ilustraci algoritmu Most parts byly převzaté od Kooi \cite{kooi}. Použití pouze pěti počátečních kódů bylo inspirováno Neuwirthem \cite{neuwirth}.

Na tuto práce lze navázat například důkladnějším prozkoumáním rozvinutí orientovaného grafu do stromu a případně obrácením operace pomocí ekvivalence na množině stavů. V našem případě by dva stavy byly ekvivalentní pokud by se jejich množiny kandidátů rovnaly. Tento postup by také bylo možné aplikovat na nějakou obecnější skupinu her založených na hádání tajného prvku množiny. Dále by šlo prozkoumat časovou složitost popisovaných deterministických algoritmů. Vzhledem k tomu, že prohledání prostoru kódů $H_{n,k}$ má časovou složitost $n^k$, pro rostoucí $n$ jsou tyto algoritmy nepoužitelné. Možná by ale šlo nalézt nějakou horní mez na počet kandidátů v $i$-té iteraci algoritmu. 


V zadání bakalářské práce byl uveden článek od autorů Berghman, Goossens, Leus \cite{BERGHMAN20091880}, popisující genetický algoritmus řešící hru Mastermind. Tento algoritmus se ale od algoritmů popisovaných v této práci výrazně liší, a proto do práce nebyl zařazen. 

