\chapter{Pravidla hry Mastermind}

\subsubsection{Značení}
V práci budeme používat následující značení:
\begin{itemize}
    \item Nechť $S$ je konečná množina. Potom $|S|$ značíme její mohutnost, tedy počet prvků.
    \item Nechť $K$ je množina. Symbolem $\mathcal{P}(K)$ značíme množinu všech podmnožin~$K$.
    \item Znakem $\bigsqcup$ budeme značit sjednocení disjunktních množin. Značení $\bigsqcup_{i\in \mathbb{N}} S_i$ implikuje, že $S_i$ jsou disjunktní. 
    \item Nechť $K_1$ a $K_2$ jsou množiny a $K_2$ je podmnožinou $K_1$. Značení $K_2 \subseteq K_1$ povoluje rovnost množin. Značení $K_2 \subset K_1$ rovnost množin zakazuje.
\end{itemize}

Zavedeme definice potřebné pro vysvětlení průběhu hry Mastermind. Nejprve vymezíme prostor, se kterým pak dále budeme pracovat. 

%množinu \[ \{ u = u_1u_2 \dots u_n \hspace{2px} \mid \hspace{2px} u_i \in \{1, 2, 3, \dots, k\}, \hspace{2px} i \in \{1, 2, 3, \dots, n\} \}\]
\begin{definice}[Prostor kódů]\label{def01:1}
  Nechť $n \in \N $, $k \in \N $. Potom \emph{prostor kódů o $n$ pozicích a $k$ barvách} definujeme jako množinu $H_{n,k} = \{1, 2, 3, \dots, k\}^n$. Prvky množiny $H_{n,k}$ zapisujeme jako posloupnosti $u = u_1u_2\dots u_n$ a nazýváme je \emph{kódy}.
\end{definice}
%Číslo $n \in \N $ budeme nazývat počet pozic a $k \in \N $ počet barev. Číslu $u_i \in \{1, 2, 3, \dots, k\}$ budeme říkat barva. 


Pro nějaké dva kódy z $H_{n,k}$ zavedeme ohodnocení bílými a černými kolíčky inspirovanou D. Knuthem \cite{donald_e__knuth_1977}. Ohodnocení dává určitou informaci o tom, jak se dva kódy podobají.


% Označíme-li dále $N = \sum_{j = 1}^k  \min(n_j, m_j)$.
% Množinu všech ohodnocení v $H_{n,k}$ značíme $S$.
\begin{definice}[Ohodnocení]\label{ohodnoceni}
  Nechť $u,v \in H_{n,k}$ jsou kódy, potom \emph{ohodnocení kódů $u$ a $v$} je uspořádaná dvojice $(b,w) \in \N_0 \times \N_0$, kde $b$ a $w$ jsou počty černých a~bílých kolíčků definované následovně. \emph{Počet černých kolíčků} značí počet prvků množiny $\{i\in\{1,\dots, n\}\mid  u_i = v_i \}$. Označme $n_j$ počet výskytů barvy $j$ v kódu~$u$, $m_j$ počet výskytů barvy $j$ v kódu $v$. \emph{Počet bílých kolíčků} je $w = \left( \sum_{j = 1}^k \min(n_j, m_j) \right) -b$.

  Dále definujeme funkci $s \colon H_{n,k} \times H_{n,k} \to \mathbb{N}_0\times \mathbb{N}_0$, která dvojici kódů přiřadí jejich ohodnocení. Množinu všech možných ohodnocení kódů v $H_{n,k}$ značíme $S_{n,k}$.
\end{definice}
\begin{pozn}
    Funkci $s$ lze interpretovat také jako funkci, která jednomu kódu přiřadí ohodnocení vzhledem k druhému kódu. Ohodnocení $u$ vzhledem ke kódu $v$ značíme $s(u,v)$. 
\end{pozn}

Počet černých kolíčků pro nějaké dva kódy značí počet pozic, na kterých mají tyto kódy shodnou barvu. Bílé kolíčky značí počet zbývajících pozic v prvním kódu, které se shodují s nějakou jinou pozicí v druhém kódu v barvě. Nejprve se~vyhodnotí počet černých kolíčků a potom až počet bílých tak, aby každá pozice v~obou kódech přispívala maximálně k jednomu kolíčku. 

% K popisu hry Mastermind se nám bude hodit definice, která kódu $u$ přiřadí ohodnocení vzhledem k neznámému kódu $v$. 

% \begin{definice}[Ohodnocení vzhledem ke kódu]
%   Nechť $v \in H_{n,k}$ je kód. Definujeme funkci
%   \begin{align*}
%         s_v \colon H_{n,k} &\to \mathbb{N}_0\times \mathbb{N}_0 \\
%         u &\mapsto s(u,v)
%   \end{align*}
%   Pro $u\in H_{n,k}$ definujeme ohodnocení $u$ vzhledem k $v$ jako $s_v(u)$.
% \end{definice}


Triviálním pozorováním je fakt, že pro libovolné dva kódy je součet černých a bílých kolíčků v jejich ohodnocení maximálně $n$. To plyne z toho, že hodnota $\sum_{j = 1}^k  \min(n_j, m_j)$ je menší nebo rovna $n$. Zároveň v prostoru $H_{n,k}$ neexistuje ohodnocení $(n-1,1)$. Pokud se totiž kódy shodují právě na $n-1$ pozicích, musí se na zbývající pozici lišit. Konstrukci kódů, které nabývají všechna možná zbývající ohodnocení uvádíme v následujícím tvrzení.
\begin{tvrz}\label{tvrzohodnoceni}
    Nechť $n \in \N$, $k \in \N, k>2$. Potom pro všechna $(b,w)\in \mathbb{N}_0 \times \mathbb{N}_0$ taková, že $b + w \leq n$ existuje dvojice kódů $u,v \in H_{n,k}$ s ohodnocením $(b,w)$.
\end{tvrz}
\begin{dukaz}

    %Ukážeme, že pro každé $b$ a $w$, $b+w \leq n$ existují dva kódy $u,v \in H_{n,k}$, takové, že $s(u,v) = (b,w)$.
    
    
    Situaci rozdělíme na dva případy podle parity $w$. Nechť napřed $w$ je sudé. Kódy $u$ a $v$ zvolíme následovně:
\[
  u = 
    \underbrace{11\dots1}_{b}
    \underbrace{11\dots1}_{\frac{w}{2}}
    \underbrace{22\dots2}_{\frac{w}{2}}
    \underbrace{11\dots1}_{n-b-w}
 \]
 \[
  v = 
    \underbrace{11\dots1}_{b}
    \underbrace{22\dots2}_{\frac{w}{2}}
    \underbrace{11\dots1}_{\frac{w}{2}}
    \underbrace{22\dots2}_{n-b-w}
 \]
Vidíme, že počet černých kolíčků kódů $u$ a $v$ je $b$. Počet bílých kolíčků $w'$ je potom
\begin{align*} 
w' &= \min\left(n - \frac{w}{2}, b + \frac{w}{2}\right) + \min\left(\frac{w}{2}, n - b - \frac{w}{2}\right) - b \\ 
&= \min\left(n - b - \frac{w}{2}, \frac{w}{2}\right) + \min\left(\frac{w}{2}, n - b - \frac{w}{2}\right) \\
&= 2\cdot\min\left(\frac{w}{2}, n - b - \frac{w}{2}\right) = w
\end{align*}
V poslední rovnosti jsme využili předpokladu, že $b+w \leq n$, a tedy $\frac{w}{2} \leq n - b - \frac{w}{2}$

Pro $w$ liché zvolíme kódy následovně:
\[
  u = 
    \underbrace{11\dots1}_{b}
    \underbrace{11\dots1}_{\frac{w-1}{2}}
    \underbrace{22\dots2}_{\frac{w-1}{2}-1}
    2
    3
    \underbrace{11\dots1}_{n-b-w}
 \]
 \[
  v = 
    \underbrace{11\dots1}_{b}
    \underbrace{22\dots2}_{\frac{w-1}{2}}
    \underbrace{11\dots1}_{\frac{w-1}{2}-1}
    3
    1
    \underbrace{22\dots2}_{n-b-w}
 \]
Počet černých kolíčků je $b$. Počet bílých kolíčků $w'$ kódů $u$ a $v$ je potom
\begin{align*} 
w' = \min\left(n-w+\frac{w-1}{2}, b + \frac{w-1}{2}\right) + \min\left(\frac{w-1}{2}, \frac{w-1}{2} + n-b-w\right) + \\ +\min(1,1) - b 
 = b + \frac{w-1}{2} + \frac{w-1}{2} + 1 - b = w
\end{align*}
\end{dukaz}

Silnější tvrzení platí pro případ se dvěma barvami. 

\begin{tvrz}\label{tvrzohodnoceni2}
    Nechť $n \in \N, k=2$. Nechť dále $u, v \in H_{n,k}$ jsou kódy s ohodnocením $(b,w)$. Potom $b + w \leq n, \hspace{5px} (b,w) \neq (n-1,1)$ a $w$ je sudé.
\end{tvrz}
\begin{dukaz}
Omezení $b + w \leq n, \hspace{3px} (b,w) \neq (n-1,1)$ platí obdobně jako v důkazu tvrzení~\ref{tvrzohodnoceni}. Kódy $u$ a $v$ lze permutací pozic dostat do následujících tvarů:
    \[
  u = 
    \underbrace{11\dots1}_{c}
    \underbrace{22\dots2}_{d}
    \underbrace{11\dots1}_{x}
    \underbrace{22\dots2}_{y}
 \]
 \[
  v = 
    \underbrace{11\dots1}_{c}
    \underbrace{22\dots2}_{d}
    \underbrace{22\dots2}_{x}
    \underbrace{11\dots1}_{y}
 \]
 kde $b = c + d$. 

 Potom
 \begin{align*}
     w &= \min(c + x, c + y) + \min(d + y, d + x) - b =\\
     &= \min(x, y) + \min(y, x) = 2\cdot \min(x, y)
 \end{align*}
% Navíc pro každé $(b,w)$ splňující podmínky tvrzení jsme schopni pomocí této konstrukce najít dva kódy, které mají ohodnocení $(b,w)$.
\end{dukaz}

    
    %Nyní dokážeme, existenci ohodnocení pro nějaké $(b,w)$. Nechť napřed $w = 0$. Zvolme prvních $b$ pozic v obou kódech na barvu $1$. Protože $k>2$, můžeme pro zbylé pozice v prvním kódu zvolit barvu $1$ a pro zbylé pozice ve druhém kódu zvolit barvu $2$. Snadným ověřením z definice zjistíme, že tyto dva kódy mají ohodnocení $(b,0)$.   Nechť nyní $w > 0$ a $b+w < n$. První kód vytvoříme takto. Položme prvních $b$ pozic na barvu $1$. Položme dále dalších $w$ pozic v prvním kódu střídavě $1212\dots$ a zbylé pozice v prvním kódu vyplníme třetí barvou. Druhý kód vyplníme následovně. Položme prvních $b$ pozic na barvu $1$. Dalších $w + 1$ pozic zvolíme jako $212121\dots$ a kód doplníme barvou dva.     Pokud $b+w = n$ a $w$ je sudé. Náš postup bude fungovat s jedinou obměnou a to, že posledních posledních $w$ pozic druhého kódu $2121\dots$. Pokud $w$ je liché, zvolíme poseldních $w$ pozic prvního kódu $1212\dots3$ a posledních $w$ pozic druhého kódu $3121\dots$.





%V příkladu níže je ukázka konstrukce kódů s daným ohodnocením.
%\begin{prikl}
 %   Pro ohodnocení $(b,w), w > 0, b+w < n$ kódy mohou vypadat například takto:
  %  \[111111,1212121,33333\]
   % \[111111,21212121,2222\]

%    Pro případ $b+w = n$ a $w$ liché lze kódy sestavit takto:
 %   \[111111,1212123\]
  %  \[111111,3121212\]
%\end{prikl}

Nyní jsme připraveni definovat hru Mastermind. 

\begin{definice}[{[n,k]-Mastermind}]
    Nechť $n \in \N $, $k \in \N$. \emph{[n,k]-Mastermind} je hra pro~dvě osoby, zadavatele a hráče, 
    probíhající podle následujícího schématu. 
    \begin{enumerate}
        \item Zadavatel vytvoří tajný kód $v \in H_{n,k}$.
        \item Hra dále probíhá po kolech $i = 1,2,3,\dots$. Každé kolo se skládá ze dvou kroků:
            \begin{enumerate}
                \item Hráč určí kód (tah, pokus) $u_i \in H_{n,k}$.
                \item Zadavatel ohodnotí pokus $u_i$ vzhledem k tajnému kódu $v$. 
            \end{enumerate}
        \item Hra končí v případě, že pokus je ohodnocen $n$ černými kolíčky (zadaný kód se shoduje s tajným kódem). 
    \end{enumerate}
    Cílem hry [n,k]-Mastermind je nalézt tajný kód na co nejmenší počet pokusů.
\end{definice}
\begin{pozn}
     V případě, že hráč ve svých tazích neopakuje kódy, hra jistě skončí, neboť počet možných kódů je omezen hodnotou $n^k$.
\end{pozn}
% vyčerpáním všech možností pro tajný kód.  

% přidat různé varianty hry mastermind??? např statická verze, hra pouze s kandidáty atd. 




