\chapter{Výsledky algoritmů}
V následující kapitole porovnáme tři popisované algoritmy. 

\section{Vlastnosti algoritmů}
Při srovnávání algoritmů budeme používat několik kritérií. Jedním z přirozených možností je sledovat \textbf{maximální počet pokusů}, které algoritmus potřebuje k nalezení jakéhokoli tajného kódu $v \in H_{n,k}$. K nalezení maximálního počtu pokusů nám tedy stačí vyzkoušet chod algoritmu na všech tajných kódech $v \in H_{n,k}$.

% jiná formulace - Co když nám ale nezáleží na maximálním počtu pokusů? 
Druhý ukazatel, který u algoritmů budeme srovnávat je \textbf{průměrný počet pokusů} na hru. Ukážeme, že nejmenší maximální počet pokusů není zárukou pro minimální průměrný počet pokusů na hru. 

Posledním kritériem, kterého si budeme u algoritmů všímat je výběr prvního tahu. Zajímá nás, jestli první tah, který algoritmus zvolí, je optimální. Tedy zda neexistuje jiný první pokus, který by s následným použitím algoritmu vracel lepší výsledky. 

\begin{definice}[Konzistentní algoritmus]\label{defkonzistentnialg}
    Uvažujme algoritmus řešící [n,k]-Mastermind. Nechť $u \in H_{n,k}$ je kód, který algoritmus zvolí jako první pokus. Řekneme, že algoritmus řešící [n,k]-Mastermind je konzistentní, pokud pro všechny kódy $u' \in H_{n,k}$ platí, že algoritmus, který místo $u$ volí jako první pokus kód $u'$ nedosahuje lepšího průměrného počtu pokusů.
\end{definice}
Konzistentnost budeme zkoumat pouze pro pět zmíněných počátečních kódů $(1111, 1112, 1122, 1123, 1234)$.
%Potom porovnáme všechny ostatní kódy se stejnou vlastnostní - např všechny kódy, které mají dvě a dvě pozice se stejnou barvou. 

\section{min-max}

Tabulka \ref{tabminmaxvysl} ukazuje výsledky Knuthova algoritmu v závislosti na volbě prvního pokusu. Byla sestrojena za pomoci programu v pythonu, který byl vytvořen pro tuto práci. \cite{Simsa_Strategies_for_Mastermind_2025} 

Maximální počet pokusů, který min-max algoritmus (začínající kódem 1122) potřebuje na prolomení tajného kódu je $5$. Průměrně pak potřebuje $4.476$ pokusů. K porovnání jsou uvedeny maximální a průměrné počty pokusů pro jiné počáteční kódy. Mohli bychom totiž očekávat, že by průměrný nebo maximální počet pokusů byl menší pro jiný počáteční kód, jak bylo znázorněno v příkladu \ref{malyprikladminmax}. Nejlepších výsledků ale algoritmus dosahuje v při volbě prvního tahu $1122$, který odpovídá přirozenému výběru prvního tahu podle definice Knuthova algoritmu. 

\begin{table}[h]
\centering
\begin{tabular}{l l l l l l}
\toprule
první pokus & 1111 & 1112 & \textbf{1122} & 1123 & 1234 \\
\midrule

maximální počet pokusů 
& 6 & 6 & \textbf{5} & 6 & 6 \\

průměrný počet pokusů 
& 5.065 & 4.556 & \textbf{4.476} & 4.4776 & 4.4776\\
\bottomrule
\multicolumn{6}{l}{\footnotesize \textit{Pozn:}
Průměrný počet pokusů je zaokrouhlen na $3$ desetinná místa}
\end{tabular}
\caption{Výsledky min-max algoritmu}\label{tabminmaxvysl}
\end{table}

\subsection{entropie}

Algoritmus s entropií vybírá jako první pokus kód $1234$. Dosahuje průměrného počtu pokusů $4.415$, jak ukazuje tabulka \ref{tabentropievysl}. Pokud by ale tento algoritmus volil počáteční kód $1123$, dosáhl by lepšího průměrného počtu pokusů, konkrétně $4.383$. Algoritmus s entropií tedy není konzistentní. 

\begin{table}[h]
\centering
\begin{tabular}{l l l l l l}
\toprule
první pokus & 1111 & 1112 & 1122 & 1123 & \textbf{1234} \\
\midrule

maximální počet pokusů 
& 6 & 6 & 6 & 6 & 6 \\

průměrný počet pokusů 
& 4.911 & 4.496 & 4.428 & \textbf{4.383} & 4.415 \\
\bottomrule
\multicolumn{6}{l}{\footnotesize \textit{Pozn:}
Průměrný počet pokusů je zaokrouhlen na $3$ desetinná místa}
\end{tabular}
\caption{Výsledky algoritmu s entropií}\label{tabentropievysl}
\end{table}



\subsection{počet částí}
Metoda popsána Barteldem Kooi je konzistentní. Volí jako první tah kód $1123$ a dosahuje průměrného počtu pokusů $4.373$. Výsledky pro další počáteční tahy znázorňuje tabulka \ref{tabcastivysl}.


\begin{table}[h]
\centering
\begin{tabular}{l l l l l l}
\toprule
první pokus & 1111 & 1112 & 1122 & \textbf{1123} & 1234 \\
\midrule

maximální počet pokusů 
& 7 & 6 & 6 & 6 & 6 \\

průměrný počet pokusů 
& 4.911 & 4.503 & 4.420 & \textbf{4.373} & 4.444\\
\bottomrule
\multicolumn{6}{l}{\footnotesize \textit{Pozn:}
Průměrný počet pokusů je zaokrouhlen na $3$ desetinná místa}
\end{tabular}
\caption{Výsledky algoritmu s počtem částí}\label{tabcastivysl}
\end{table}


\subsection{Očekávaný počet}
Případná čtvrtá strategie...
\begin{table}[h]
\centering
\begin{tabular}{l l l l l l}
\toprule
první pokus & 1111 & 1112 & 1122 & 1123 & 1234 \\
\midrule

maximální počet pokusů 
& 6 & 6 & 5 & 6 & 6 \\

průměrný počet pokusů 
& 4.911 & 4.529 & 4.453 & 4.394 & 4.434 \\
\bottomrule
\multicolumn{6}{l}{\footnotesize \textit{Pozn:}
Průměrný počet pokusů je zaokrouhlen na $3$ desetinná místa}
\end{tabular}
\caption{Výsledky algoritmu s očekávaným počtem kandidátů}\label{tabminmaxvysl}
\end{table}


\section{Diskuze}
Poznámky:
\begin{itemize}
    \item Nejlepší maximální počet pokusů v případě min-max algoritmu dává smysl, protože algoritmus minimalizuje maximální počet kandidátů v tabulce rozdělení. 
    \item Zároveň fakt, že zbylé algoritmy dosahují lepšího průměrného počtu pokusů také odpovídá tomu, že sledují nějakou očekávanou hodnotu v tabulce rozdělení. - očekávaná hodnota počtu otázek v případě entropie, pravděpodobnost, že v následujícím tahu uhodneme tajný kód v případě počtu částí. 
    \item Algoritmy, které by zkoumaly tabulky rozdělení dva kroky dopředu by jistě byly lepší, výpočetně by to ale bylo mnohem náročnější.
    \item šlo by najít nějakou variantu zmíněných algoritmů, která by řešila mastermind na 5 tahů s výrazně lepším průměrem? - Knuth něco na tenhle způsob ukazoval, ale možná jen v případě jeho algoritmu.
    \item Volba prvního tahu, který by obsahoval vyšší barvy by mohla být lepší, protože algoritmus volí lexikograficky menší kódy. Podobně by bylo možné náhodně promíchat seřazení všech kódů. 
    \item Čím je strategie optimálnější, tím hůře jde popsat.
    \item Strategii optimalizující průměrný počet pokusů nalezl v roce 1993 K. Koyama \cite{koyama}. Ukázal, že lze dosáhnout strategie s průměrným počtem pokusů $4.340$.
\end{itemize}




