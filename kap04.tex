\chapter{Výsledky algoritmů}
V následující kapitole porovnáme tři popisované algoritmy. 

\section{Vlastnosti algoritmů}
Při srovnávání algoritmů budeme používat několik kritérií. Jedním z přirozených možností je sledovat maximální počet pokusů, které algoritmus potřebuje k nalezení jakéhokoli tajného kódu $v \in H_{n,k}$. K nalezení maximálního počtu pokusů nám tedy stačí vyzkoušet chod algoritmu na všech tajných kódech $v \in H_{n,k}$. 
Druhá možnost je sestrojit strom algoritmu a nalézt maximální délku cesty od listu ke kořeni.

\begin{definice}[Maximální počet pokusů]\label{defmaxpocetpokusu}
    Nechť $T$ je strom algoritmu pro nějaký algoritmus řešící hru [n,k]-Mastermind. Maximální počet pokusů tohoto algoritmu definujeme jako nejdelší délku cesty od kořene $T$ k nějakému listu $T$. 
\end{definice}

Druhý ukazatel, který u algoritmů budeme srovnávat je průměrný počet pokusů na hru. Ten lze opět vyčíst ze stromu algoritmu.

\begin{definice}[Průměrný počet pokusů]\label{defprumpocetpokusu}
    Nechť $T = (V,E)$ je strom algoritmu pro nějaký algoritmus řešící hru [n,k]-Mastermind. Pro list $P$ označíme $d(P)$ délku cesty od kořene k tomuto listu. Průměrný počet pokusů algoritmu definujeme jako 
    \[\frac{1}{k^n}\sum_{P\in V, P \text{ list}} d(P).\]
\end{definice}

Posledním kritériem, kterého si budeme u algoritmů všímat je výběr prvního tahu. Zajímá nás, jestli první tah, který algoritmus zvolí, je optimální. Tedy zda neexistuje jiný první pokus, který by s následným použitím algoritmu vracel lepší výsledky. Tuto vlastnost budeme pojmenovávat \textbf{konzistence}.

Konzistenci budeme zkoumat pro pět počátečních kódů $(1111, 1112, 1122, 1123, 1234)$. Obhajujeme to tím, že neexistuje kód, který by vytvořil nějakou jinou valuaci v prvním tahu než tyto kódy. To plyne z toho, že každý kód vznikne nějakou substitucí barev a permutací pozic těchto pěti kódů. Tyto operace nezmění velikosti potomků ${H_{4,6}}_{u,r}$, protože jednotlivé pozice a barvy jsou rovnoměrně rozloženy v celém $H_{4,6}$. A protože tři použité valuace vycházejí právě z velikostí potomků, jejich hodnoty se nezmění. 

V sekci $4.5$ diskutujeme možný vliv výběru lexikograficky vyššího prvního tahu z důvodu následného výběru lexikograficky minimálních kódů ve všech algoritmech. V této práci se ale dále touto problematikou zabývat nebudeme. 

V tabulce \ref{tabvaluaceprvnichtahu} jsou zobrazeny valuace těchto kódů vzhledem k $H_{4,6}$. Nejlepší valuace prvního pokusu podle zvolené strategie v konkrétním algoritmu je zobrazena tučně. 


\begin{table}[h]
\centering
\begin{tabular}{l l l l l l}
\toprule
kód & 1111 & 1112 & 1122 & 1123 & 1234 \\
\midrule

maximum & 625 & 317 & \textbf{256} & 276 & 312 \\
entropie & 1.50 & 2.69 & 2.89 & 3.04 & \textbf{3.06}\\

% [1.4984350761704437, 2.6934339172716406, 2.885102174947102, 3.0436979819559213, 3.0566709153318907]

%vážený průměr - $\sum_{i = 1}^{14} |S_i|  \cdot \pr (S_i)$ & 512.0 & 235.9 & 204.5 & \textbf{185.3} & 188.2 \\

% [511.9799382716049, 235.94907407407408, 204.53549382716048, 185.26851851851853, 188.18981481481484]

počet částí & 5 & 11 & 13 & \textbf{14} & \textbf{14} \\

\bottomrule
\multicolumn{6}{l}{\footnotesize \textit{Pozn:}
Entropie je zaokrouhlena na dvě desetinná místa.
%, vážený průměr na jedno.
}

\end{tabular}
\caption{Hodnoty valuace kódů pro první tah}
\label{tabvaluaceprvnichtahu}
\end{table}



\section{Způsob testování}
Výsledky algoritmů byly nalezeny vytvořením stromu algoritmu pro každou metodu. Strom byl sestavován postupně, kdy pro každý stav (vrchol na grafu) byl nalezen nejlepší další pokus. Ve chvíli, kdy program dojde do listu hranou $(n,0)$, je uložena aktuální vzdálenost listu od kořene (počet pokusů potřebný k nalezení daného tajného kódu) do seznamu s počty pokusů. 
%V průběhu vytváření stromu byly zjišťovány vzdálenosti konečných stavů (listů) od kořene odpovídající počtu pokusů potřebných k uhádnutí daného kódu. 
Následně z těchto hodnot byly vypočteny hodnoty Maximální počet pokusů a Průměrný počet pokusů. Konzistence algoritmů byla nalezena za pomocí pevného určení prvního pokusu při vytváření stromu algoritmů. 

Tento postup byl implementován v jazyce Python pro účel této práce. Zdrojový kód a jeho dokumentace je dostupný na GitHub \cite{Simsa_Strategies_for_Mastermind_2025}. 



% Výsledky algoritmů byly nalezeny pomocí programu napsaného v jazyce Python pro účel této práce \cite{Simsa_Strategies_for_Mastermind_2025}. Maximální a průměrné počty pokusů byly nalezeny za pomocí sestavení stromu řešení [4,6]-Mastermind jako na obrázku \ref{fig22minmax}. Strom byl sestavován postupně, kdy pro každý stav (vrchol na grafu) byl nalezen nejlepší další pokus. Ve chvíli, kdy program dojde do listu hranou $(n,0)$, uchová si aktuální počet pokusů potřebný k nalezení daného tajného kódu. Program končí ve chvíli, kdy je každý kód v listu. Nakonec se spočítá očekávaný a maximální počet pokusů ze zachovaných hodnot pro jednotlivé kódy.

% Existují dvě situace, pro které se stav dostane do listu. První je v případě, kdy stav je složen z více kódů a tedy nevíme, který z nich je tajný kódem. V případě, že algoritmus vybere jako další pokus kandidáta, může se stát, že tento kandidát byl dokonce tajným kódem. V tu chvíli máme štěstí a nalezli jsme počet pokusů, který potřebuje min-max algoritmus k nalezení tohoto tajného kódu. Druhá situace nastává, kdy stavem je jednoprvková množina s pouze jedním kódem. Zároveň ho min-max algoritmus zatím nezahrál. V tuto chvíli algoritmus ví, který kód je tajným, ale musí ho ještě zahrát, protože hra končí ve chvíli, kdy je zahraný tajný kód. 

\section{min-max}

Tabulka \ref{tabminmaxvysl} ukazuje výsledky Knuthova algoritmu v závislosti na volbě prvního pokusu. Byla sestrojena za pomoci programu v pythonu, který byl vytvořen pro tuto práci. \cite{Simsa_Strategies_for_Mastermind_2025} 

Maximální počet pokusů, který min-max algoritmus (začínající kódem 1122) potřebuje na prolomení tajného kódu je $5$. Průměrně pak potřebuje $4.476$ pokusů. K porovnání jsou uvedeny maximální a průměrné počty pokusů pro jiné počáteční kódy. Mohli bychom totiž očekávat, že by průměrný nebo maximální počet pokusů byl menší pro jiný počáteční kód, jak bylo znázorněno v příkladu \ref{malyprikladminmax}. Nejlepších výsledků ale algoritmus dosahuje v při volbě prvního tahu $1122$, který odpovídá přirozenému výběru prvního tahu podle definice Knuthova algoritmu. 

\begin{table}[h]
\centering
\begin{tabular}{l l l l l l}
\toprule
první pokus & 1111 & 1112 & \textbf{1122} & 1123 & 1234 \\
\midrule

maximální počet pokusů 
& 6 & 6 & \textbf{5} & 6 & 6 \\

průměrný počet pokusů 
& 5.065 & 4.556 & \textbf{4.476} & 4.4776 & 4.4776\\
\bottomrule
\multicolumn{6}{l}{\footnotesize \textit{Pozn:}
Průměrný počet pokusů je zaokrouhlen na $3$ desetinná místa}
\end{tabular}
\caption{Výsledky min-max algoritmu}\label{tabminmaxvysl}
\end{table}

\subsection{Max entropy}

Algoritmus Max entropy vybírá jako první pokus kód $1234$. Dosahuje průměrného počtu pokusů $4.415$, jak ukazuje tabulka \ref{tabentropievysl}. Pokud by ale tento algoritmus volil počáteční kód $1123$, dosáhl by lepšího průměrného počtu pokusů, konkrétně $4.383$. Algoritmus Max entropy tedy není konzistentní. 

\begin{table}[h]
\centering
\begin{tabular}{l l l l l l}
\toprule
první pokus & 1111 & 1112 & 1122 & 1123 & \textbf{1234} \\
\midrule

maximální počet pokusů 
& 6 & 6 & 6 & 6 & 6 \\

průměrný počet pokusů 
& 4.911 & 4.496 & 4.428 & \textbf{4.383} & 4.415 \\
\bottomrule
\multicolumn{6}{l}{\footnotesize \textit{Pozn:}
Průměrný počet pokusů je zaokrouhlen na $3$ desetinná místa}
\end{tabular}
\caption{Výsledky algoritmu s entropií}\label{tabentropievysl}
\end{table}



\subsection{Most parts}
Metoda popsána Barteldem Kooi je konzistentní. Volí jako první tah kód $1123$ a dosahuje průměrného počtu pokusů $4.373$. Výsledky pro další počáteční tahy znázorňuje tabulka \ref{tabcastivysl}.


\begin{table}[h]
\centering
\begin{tabular}{l l l l l l}
\toprule
první pokus & 1111 & 1112 & 1122 & \textbf{1123} & 1234 \\
\midrule

maximální počet pokusů 
& 7 & 6 & 6 & 6 & 6 \\

průměrný počet pokusů 
& 4.911 & 4.503 & 4.420 & \textbf{4.373} & 4.444\\
\bottomrule
\multicolumn{6}{l}{\footnotesize \textit{Pozn:}
Průměrný počet pokusů je zaokrouhlen na $3$ desetinná místa}
\end{tabular}
\caption{Výsledky algoritmu s počtem částí}\label{tabcastivysl}
\end{table}


\subsection{Očekávaný počet}
Případná čtvrtá strategie...
\begin{table}[h]
\centering
\begin{tabular}{l l l l l l}
\toprule
první pokus & 1111 & 1112 & 1122 & 1123 & 1234 \\
\midrule

maximální počet pokusů 
& 6 & 6 & 5 & 6 & 6 \\

průměrný počet pokusů 
& 4.911 & 4.529 & 4.453 & 4.394 & 4.434 \\
\bottomrule
\multicolumn{6}{l}{\footnotesize \textit{Pozn:}
Průměrný počet pokusů je zaokrouhlen na $3$ desetinná místa}
\end{tabular}
\caption{Výsledky algoritmu s očekávaným počtem kandidátů}\label{tabminmaxvysl}
\end{table}

\section{Varianty/zlepšení algoritmů}
\subsubsection{Volba pokusů pouze z kandidátů}
Všechny algoritmy uvedené do této chvíle vybíraly následující pokus ze všech kódů $u\in H_{n,k}$. Předpokládejme, že se nacházíme ve stavu $P$ s množinou kandidátů na tajný kód $K$. Pokud zahrajeme kód, který nenáleží do množiny $K$, nemáme šanci tajný kód v tomto kole prolomit. Logickou variantou algoritmů je způsob, který povoluje výběr dalšího pokusu pouze z množiny kandidátů stavu. V základním algoritmu \ref{alg-default} by množina $U$ na řádce $8$ měla tvar
\[U = \{u \in K \mid f_K(u) = F(f_K)\}.\]
Níže uvádíme výsledky algoritmů volící pouze kandidáty aplikované na [4,6]-Mastermind. Výsledky byly dosaženy stejným postupem jako u algoritmů výše, tedy postupným sestrojením stromu algoritmů. 

\begin{table}[h]
\centering
\begin{tabular}{l l l l}
\toprule
první pokus & Min-max & Max entropy & Počet potomků  \\
\midrule

maximální počet pokusů 
& 6 & 6 & 7  \\

průměrný počet pokusů 
& 4.497 & 4.465 & 4.399 \\
\bottomrule
\multicolumn{4}{l}{\footnotesize \textit{Pozn:}
Průměrný počet pokusů je zaokrouhlen na $3$ desetinná místa}
\end{tabular}
\caption{Výsledky algoritmů volící pouze kandidáty}\label{tabminmaxvysl}
\end{table}

Průměrný počet pokusů se zhoršil ve všech algoritmech, v algoritmu min-max a Max parts ze zhoršil i maximální počet pokusů. Tato volba algoritmů ale zmenší časovou složitost algoritmů. 

\begin{tvrz}
    Uvažujme algoritmus \ref{alg-default} s výběrem množiny $U$ na řádku $7$ podle následujícího předpisu.
    \[U = \{u \in K \mid f_K(u) = F(f_K)\}\]
    Potom tento algoritmus má časovou složitost 
     \[O \left( \sum_{j = 1}^z  m_j^2 \cdot n^2 \cdot \frac{n^2 + 3n}{2}\right)\]
\end{tvrz}
\begin{dukaz}
    Tvrzení plyne z tvrzení ...
\end{dukaz}

\subsubsection{Jiná volba prvního tahu}



\section{Diskuze}



Poznámky:
\begin{itemize}
    \item Nejlepší maximální počet pokusů v případě min-max algoritmu dává smysl, protože algoritmus minimalizuje maximální počet kandidátů v tabulce rozdělení. 
    \item Zároveň fakt, že zbylé algoritmy dosahují lepšího průměrného počtu pokusů také odpovídá tomu, že sledují nějakou očekávanou hodnotu v tabulce rozdělení. - očekávaná hodnota počtu otázek v případě entropie, pravděpodobnost, že v následujícím tahu uhodneme tajný kód v případě počtu částí. 
    \item Algoritmy, které by zkoumaly tabulky rozdělení dva kroky dopředu by jistě byly lepší, výpočetně by to ale bylo mnohem náročnější.
    \item šlo by najít nějakou variantu zmíněných algoritmů, která by řešila mastermind na 5 tahů s výrazně lepším průměrem? - Knuth něco na tenhle způsob ukazoval, ale možná jen v případě jeho algoritmu.
    \item Volba prvního tahu, který by obsahoval vyšší barvy by mohla být lepší, protože algoritmus volí lexikograficky menší kódy. Podobně by bylo možné náhodně promíchat seřazení všech kódů. 
    \item Čím je strategie optimálnější, tím hůře jde popsat.
    \item Strategii optimalizující průměrný počet pokusů nalezl v roce 1993 K. Koyama \cite{koyama}. Ukázal, že lze dosáhnout strategie s průměrným počtem pokusů $4.340$.
    \item aplikace problému mastermind jsou diskutovány v článku doerr playing mastermind with many colours na straně 4
    \item odhad s entropií nefunguje tak dobře. Mohlo by to být silně nerovnoměrně rozdělenými kódy. 
\end{itemize}




