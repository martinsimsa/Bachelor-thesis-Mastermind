\chapter{Výsledky}
% V následující kapitole porovnáme tři popisované algoritmy. 

\section{Vlastnosti algoritmů}
Při srovnávání algoritmů budeme používat několik kritérií. První hodnotou je maximální počet pokusů, které algoritmus potřebuje k nalezení libovolného tajného kódu $v \in H_{n,k}$. K nalezení maximálního počtu pokusů stačí vyzkoušet chod algoritmu na všech tajných kódech $v \in H_{n,k}$. V sekci se stromem algoritmu jsme ukázali, že tato hodnota je ekvivalentní největší vzdálenosti listu od kořene ve stromě algoritmu ve smyslu počtu hran.

\begin{definice}[Maximální počet pokusů]\label{defmaxpocetpokusu}
    Nechť $T$ je strom algoritmu pro nějaký algoritmus řešící hru [n,k]-Mastermind. \emph{Maximální počet pokusů} tohoto algoritmu definujeme jako délku nejdelší cesty (počet jejích hran) od kořene $T$ k nějakému listu $T$. 
\end{definice}

Druhý ukazatel, který u algoritmů budeme srovnávat, je průměrný počet pokusů na hru. Ten lze opět odvodit ze stromu algoritmu.

\begin{definice}[Průměrný počet pokusů]\label{defprumpocetpokusu}
    Nechť $T = (V,E)$ je strom algoritmu pro~nějaký algoritmus řešící hru [n,k]-Mastermind. Pro list $A$ označíme $d(A)$ jeho vzdálenost od kořene ve smyslu počtu hran. \emph{Průměrný počet pokusů} algoritmu definujeme jako hodnotu
    \[\frac{1}{k^n}\sum_{A\in V, A \text{ list}} d(A).\]
\end{definice}

Posledním kritériem, kterého si budeme u algoritmů všímat je výběr prvního tahu. Zajímá nás, jestli první tah, který algoritmus zvolí, je optimální. Tedy jestli neexistuje jiný první pokus, který by s následným použitím algoritmu vracel lepší výsledky. Tuto vlastnost pojmenujeme \emph{konzistence}. V případě [4,6]-Mastermindu budeme konzistenci zkoumat pro pět počátečních kódů $1111$, $1112$, $1122$, $1123$, $1234$. Obhajujeme to tím, že neexistuje kód, který by vytvořil nějakou jinou valuaci v prvním tahu než tyto kódy. Tato myšlenka byla převzatá od Neuwirtha \cite{neuwirth}. Každý kód totiž vznikne nějakou substitucí barev a permutací pozic těchto pěti kódů. Tyto operace nezmění velikosti potomků ${H_{4,6}}_{u,r}$, protože jednotlivé pozice a barvy jsou rovnoměrně rozloženy v celém prostoru kódů. A protože tři použité valuace vycházejí právě z velikostí potomků, jejich hodnoty se nezmění. 

% V sekci $4.5$ diskutujeme možný vliv výběru lexikograficky vyššího prvního tahu z důvodu následného výběru lexikograficky minimálních kódů ve všech algoritmech.

V tabulce \ref{tabvaluaceprvnichtahu} jsou zobrazeny valuace pěti zmíněných kódů vzhledem k $H_{4,6}$. Nejlepší valuace prvního pokusu podle zvolené strategie v konkrétním algoritmu je zobrazena tučně. 


\begin{table}[h]
\centering
\begin{tabular}{l l l l l l}
\toprule
kód & 1111 & 1112 & 1122 & 1123 & 1234 \\
\midrule

Maximum potomka & 625 & 317 & \textbf{256} & 276 & 312 \\
Entropie potomků & 1.50 & 2.69 & 2.89 & 3.04 & \textbf{3.06}\\

% [1.4984350761704437, 2.6934339172716406, 2.885102174947102, 3.0436979819559213, 3.0566709153318907]

%vážený průměr - $\sum_{i = 1}^{14} |S_i|  \cdot \pr (S_i)$ & 512.0 & 235.9 & 204.5 & \textbf{185.3} & 188.2 \\

% [511.9799382716049, 235.94907407407408, 204.53549382716048, 185.26851851851853, 188.18981481481484]

Počet potomků & 5 & 11 & 13 & \textbf{14} & \textbf{14} \\

\bottomrule
\multicolumn{6}{l}{\footnotesize \textit{Pozn:}
Entropie je zaokrouhlena na dvě desetinná místa.
%, vážený průměr na jedno.
}

\end{tabular}
\caption{Hodnoty valuace kódů pro první tah}
\label{tabvaluaceprvnichtahu}
\end{table}



\section{Způsob testování}
Výsledky algoritmů byly nalezeny vytvořením stromu algoritmu pro každou metodu. Strom byl sestavován postupně, kdy pro každý stav (vrchol na grafu) byl nalezen nejlepší další pokus podle zvolené valuace a strategie. Ve chvíli, kdy~program dojde do listu hranou $(n,0)$, je uložena aktuální vzdálenost listu od~kořene (počet pokusů potřebný k nalezení daného tajného kódu) do seznamu s~počty pokusů. 
%V průběhu vytváření stromu byly zjišťovány vzdálenosti konečných stavů (listů) od kořene odpovídající počtu pokusů potřebných k uhádnutí daného kódu. 
Následně z těchto hodnot byly vypočten maximální a průměrný počet pokusů. Konzistence algoritmů byla nalezena sestrojením stromu algoritmu s~pevně zvoleným prvním tahem.

Tento postup byl implementován v jazyce Python pro účel této práce. Zdrojový kód je k práci přiložen v elektronické podobě (viz Příloha \ref{priloha-program}). Zároveň je přiložena i jeho dokumentace (viz Příloha \ref{priloha-dokumentace}). Dále je program dostupný na~GitHub \cite{Simsa_Strategies_for_Mastermind_2025}. 



% Výsledky algoritmů byly nalezeny pomocí programu napsaného v jazyce Python pro účel této práce \cite{Simsa_Strategies_for_Mastermind_2025}. Maximální a průměrné počty pokusů byly nalezeny za pomocí sestavení stromu řešení [4,6]-Mastermind jako na obrázku \ref{fig22minmax}. Strom byl sestavován postupně, kdy pro každý stav (vrchol na grafu) byl nalezen nejlepší další pokus. Ve chvíli, kdy program dojde do listu hranou $(n,0)$, uchová si aktuální počet pokusů potřebný k nalezení daného tajného kódu. Program končí ve chvíli, kdy je každý kód v listu. Nakonec se spočítá očekávaný a maximální počet pokusů ze zachovaných hodnot pro jednotlivé kódy.

% Existují dvě situace, pro které se stav dostane do listu. První je v případě, kdy stav je složen z více kódů a tedy nevíme, který z nich je tajný kódem. V případě, že algoritmus vybere jako další pokus kandidáta, může se stát, že tento kandidát byl dokonce tajným kódem. V tu chvíli máme štěstí a nalezli jsme počet pokusů, který potřebuje min-max algoritmus k nalezení tohoto tajného kódu. Druhá situace nastává, kdy stavem je jednoprvková množina s pouze jedním kódem. Zároveň ho min-max algoritmus zatím nezahrál. V tuto chvíli algoritmus ví, který kód je tajným, ale musí ho ještě zahrát, protože hra končí ve chvíli, kdy je zahraný tajný kód. 

\section{Výsledky algoritmů}

\subsubsection{Min-max}

Tabulka \ref{tabminmaxvysl} ukazuje výsledky Knuthova algoritmu v závislosti na volbě prvního pokusu. Min-max algoritmus jako první tah volí kód $1122$ díky nejlepší valuaci (viz tabulka \ref{tabvaluaceprvnichtahu}). Maximální počet pokusů Min-max algoritmu je $5$, a tedy je v~tomto směru optimální. Průměrně pak potřebuje $4.476$ pokusů. Při volbě jiného prvního tahu jsou ve všech případech oba výsledky horší. Min-max algoritmus je tedy konzistentní.

% Maximální počet pokusů, který min-max algoritmus (začínající kódem 1122) potřebuje na prolomení tajného kódu je $5$. Průměrně pak potřebuje $4.476$ pokusů. K porovnání jsou uvedeny maximální a průměrné počty pokusů pro jiné počáteční kódy. Mohli bychom totiž očekávat, že by průměrný nebo maximální počet pokusů byl menší pro jiný počáteční kód, jak bylo znázorněno v příkladu \ref{malyprikladminmax}. Nejlepších výsledků ale algoritmus dosahuje v při volbě prvního tahu $1122$, který odpovídá přirozenému výběru prvního tahu podle definice Knuthova algoritmu. 

\begin{table}[h]
\centering
\begin{tabular}{l l l l l l}
\toprule
První pokus & 1111 & 1112 & \textbf{1122} & 1123 & 1234 \\
\midrule

Maximální počet pokusů 
& 6 & 6 & \textbf{5} & 6 & 6 \\

Průměrný počet pokusů 
& 5.065 & 4.556 & \textbf{4.476} & 4.478 & 4.478\\
\bottomrule
\multicolumn{6}{l}{\footnotesize \textit{Pozn:}
Průměrný počet pokusů je zaokrouhlen na tři desetinná místa}
\end{tabular}
\caption{Výsledky Min-max algoritmu}\label{tabminmaxvysl}
\end{table}

\subsubsection{Max entropy}
Algoritmus Max entropy vybírá jako první pokus kód $1234$, protože tento kód dosahuje největší valuace entropie (viz tabulka \ref{tabvaluaceprvnichtahu}). Tabulka \ref{tabentropievysl} zobrazuje výsledky algoritmu Max entropy. Dosahuje průměrného počtu pokusů $4.415$, což je lepší výsledek než v případě Min-max algoritmu. Tento algoritmus ale nedokáže zaručit vítězství na pět pokusů. Navíc, pokud by tento algoritmus volil počáteční kód $1123$, dosáhl by lepšího průměrného počtu pokusů, konkrétně $4.383$. Algoritmus Max entropy tedy není konzistentní. 

% Entropie odhaduje očekávaný počet otázek při daného kódu jako dalšího pokusu za předpokladu, že by následně algoritmus dokázal nalézt tajný kód na $\log_{S_{n,k}} K$

% Vysvětlením by mohl být fakt, že entropie neodhaduje ideálně počet pokusů. Mohlo by to být tím, že rozdělení množiny $K$ do potomků $K_{u,r}$ není rovnoměrné. Entropie tedy nedokáže správně odhadnout rozložení množin kandidátů o dva tahy dál. 
%Kdybychom uvažovali valuace, které by srovnávali velikosti potomků do hloubky $2$. 
\begin{table}[h]
\centering
\begin{tabular}{l l l l l l}
\toprule
První pokus & 1111 & 1112 & 1122 & 1123 & \textbf{1234} \\
\midrule

Maximální počet pokusů 
& 6 & 6 & 6 & 6 & 6 \\

Průměrný počet pokusů 
& 4.911 & 4.494 & 4.428 & \textbf{4.383} & 4.415 \\
\bottomrule
\multicolumn{6}{l}{\footnotesize \textit{Pozn:}
Průměrný počet pokusů je zaokrouhlen na tři desetinná místa}
\end{tabular}
\caption{Výsledky algoritmu Max entropy}\label{tabentropievysl}
\end{table}



\subsubsection{Most parts}
Algoritmus Most parts volí jako první tah kód $1123$ (viz tabulka \ref{tabvaluaceprvnichtahu}) a~dosahuje průměrného počtu pokusů $4.373$. Tento výsledek, zobrazený v tabulce \ref{tabcastivysl}, je ze~všech porovnávaných algoritmů nejlepší. Maximální počet pokusů je ale~opět horší než u algoritmu Min-max, a to 6. Most parts algoritmus je konzistentní, protože průměrné počty pokusů při odlišném prvním tahu jsou vyšší.


\begin{table}[h]
\centering
\begin{tabular}{l l l l l l}
\toprule
První pokus & 1111 & 1112 & 1122 & \textbf{1123} & 1234 \\
\midrule

Maximální počet pokusů 
& 7 & 6 & 6 & 6 & 6 \\

Průměrný počet pokusů 
& 4.911 & 4.503 & 4.420 & \textbf{4.373} & 4.444\\
\bottomrule
\multicolumn{6}{l}{\footnotesize \textit{Pozn:}
Průměrný počet pokusů je zaokrouhlen na tři desetinná místa}
\end{tabular}
\caption{Výsledky algoritmu Most parts}\label{tabcastivysl}
\end{table}


% \subsection{Očekávaný počet}
% Případná čtvrtá strategie...
% \begin{table}[h]
% \centering
% \begin{tabular}{l l l l l l}
% \toprule
% první pokus & 1111 & 1112 & 1122 & 1123 & 1234 \\
% \midrule

% maximální počet pokusů 
% & 6 & 6 & 5 & 6 & 6 \\

% průměrný počet pokusů 
% & 4.911 & 4.529 & 4.453 & 4.394 & 4.434 \\
% \bottomrule
% \multicolumn{6}{l}{\footnotesize \textit{Pozn:}
% Průměrný počet pokusů je zaokrouhlen na $3$ desetinná místa}
% \end{tabular}
% \caption{Výsledky algoritmu s očekávaným počtem kandidátů}\label{tabminmaxvysl}
% \end{table}

\subsubsection{Srovnání algoritmů}
Nejlepší maximální počet pokusů v případě Min-max algoritmu dává smysl, protože algoritmus minimalizuje maximální počet kódů v potomcích. Algoritmus Most parts dosáhl ze všech zmíněných algoritmů nejlepšího průměrného počtu pokusů. B. Kooi \cite{kooi} srovnává svůj algoritmus Most parts s Min-max algoritmem z~hlediska jejich stromů algoritmů následovně. Min-max algoritmus zkoumá pouze hloubku stromu ve smyslu maximálního počtu kandidátů. Most parts algoritmus naopak hledá strom, který bude nejširší z hlediska počtu neprázdných pokračování. 

Algoritmus Max entropy dosahuje, možná překvapivě, o dost horšího průměrného počtu pokusů v porovnání s algoritmem Most parts. Je možné, že to je kvůli nešťastné volbě prvního tahu v případě čtyř pozic a šesti barev. G. Ville \cite{ville2013optimalmastermind47strategy} totiž ukázal výsledky zmíněných algoritmů v [4,7] a [5,8]-Mastermindu, kde Max entropy dosahuje naopak nejlepšího průměrného počtu pokusů. 




% Výsledek algoritmu Most parts je do jisté míry překvapivý. Most parts algoritmus totiž maximalizuje pravděpodobnost, že tajný kód v dalším kole uhodneme a nijak se nesnaží odhadnout očekávaný počet pokusů. Domnívám se, že 

% Zároveň fakt, že zbylé algoritmy dosahují lepšího průměrného počtu pokusů také odpovídá tomu, že optimalizují nějakou očekávanou hodnotu následujících stavů. V případě algoritmu Max entropy to je očekávaná hodnota počtu otázek. 

\section{Diskuze}
\subsubsection{Volba pokusů pouze z kandidátů}
Všechny algoritmy uvedené do této chvíle vybíraly následující pokus ze všech kódů $u\in H_{n,k}$. Předpokládejme, že se nacházíme ve stavu $A$ s množinou kandidátů na tajný kód $K$. Pokud zahrajeme kód, který nenáleží do množiny $K$, nemáme šanci tajný kód v tomto kole prolomit. Logickou variantou algoritmů je způsob, který povoluje výběr dalšího pokusu pouze z množiny kandidátů. V základním algoritmu \ref{alg-default} by množina $U$ na řádce $7$ měla tvar
\[U = \{u \in K \mid f_K(u) = F(f_K)\}.\]
Níže uvádíme výsledky algoritmů volící pouze kandidáty aplikované na [4,6]-Mastermind (viz tabulka \ref{tabpouzekandidati}). Výsledky byly dosaženy stejným postupem jako u~algoritmů výše, tedy postupným sestrojením stromu algoritmu. 

\begin{table}[h]
\centering
\begin{tabular}{l l l l}
\toprule
 & Min-max & Max entropy & Most parts  \\
\midrule
První pokus 
& 1122 & 1234 & 1123  \\

Maximální počet pokusů 
& 6 & 6 & 7  \\

Průměrný počet pokusů 
& 4.497 & 4.465 & 4.399 \\
\bottomrule
\multicolumn{4}{l}{\footnotesize \textit{Pozn:}
Průměrný počet pokusů je zaokrouhlen na tři desetinná místa.}
\end{tabular}
\caption{Výsledky algoritmů volící pouze kandidáty}\label{tabpouzekandidati}
\end{table}

Průměrný počet pokusů se zhoršil ve všech algoritmech, v algoritmu Min-max a Most parts ze zhoršil i maximální počet pokusů. Tato volba algoritmů ale~zmenší časovou složitost algoritmů. To plyne z toho, že algoritmus v každém stavu neporovnává všechny kódy, ale pouze kandidáty. 

% \begin{tvrz}
%     Uvažujme algoritmus \ref{alg-default} s výběrem množiny $U$ na řádku $7$ podle následujícího předpisu.
%     \[U = \{u \in K \mid f_K(u) = F(f_K)\}\]
%     Potom tento algoritmus má časovou složitost 
%      \[O \left( \sum_{j = 1}^z  m_j^2 \cdot n^2 \cdot \frac{n^2 + 3n}{2}\right)\]
% \end{tvrz}
% \begin{dukaz}
%     Tvrzení plyne z tvrzení ...
% \end{dukaz}

\subsubsection{Vícekrokové algoritmy}
Všechny popisované algoritmy používaly valuace, které zkoumají rozložení kandidátů u následníků daného stavu. 
% koumali tedy stavy o maximálně jednu hranu v grafu [n,k]-Mastermindu dál. 
Neberou tedy v potaz rozdělení následníků o dvě hrany dál, které by mohlo být silně nerovnoměrné. Intuice napovídá, že pokud bychom zkoumali stavy o dvě hrany dál, algoritmy by měli lepší výsledky. Časová složitost těchto algoritmů by ale prudce rostla. Všech kódů v $H_{n,k}$ je $k^n$, projít dvě vrstvy by tedy trvalo $k^{2n}$. 

Čím hlouběji bychom prohledávali strom [n,k]-Mastermindu, tím více bychom se blížili optimální metodě nalezené K. Koyamou a T. Laiem v roce 1993 \cite{koyama}. Jejich metoda dosahuje průměrného počtu pokusů $4.340$ s maximálním počtem pokusů $6$. V případě omezení maximálního počtu pokusů na $5$ dosáhli průměrného počtu pokusů $4.341$. 


\subsubsection{Seřazení kódů}
Zajímavým problémem je jistě výběr následujícího pokusu v případě, kdy má více kódů nejlepší valuaci. My jsme zvolili výběr lexikograficky nejmenšího kódu, protože ho použil Knuth ve svém článku \cite{donald_e__knuth_1977}. Výběr lexikograficky nejvyššího kódu například porovnává v diskuzi u svého článku Kooi \cite{kooi}. Změna výsledků je~ale~minimální. 

Také lze diskutovat, zda volba prvního kódu, který by byl lexikograficky větší, by následně s výběrem lexikograficky menších kódů byl přínosný. Ukazuje se, že pouze velikost prvního kódu nerozhoduje. Při zvolení lexikograficky největších prvních kódů nedostaneme lepší výsledek (viz tabulka \ref{tabvysokyprvnipokus}).


\begin{table}[H]
\centering
\begin{tabular}{l l l l}
\toprule
 & Min-max & Max entropy & Počet potomků  \\
\midrule

První pokus 
& 6655 & 6543 & 6654  \\


Maximální počet pokusů 
& 6 & 6 & 6  \\

Průměrný počet pokusů 
& 4.478 & 4.416 & 4.392 \\
\bottomrule
\multicolumn{4}{l}{\footnotesize \textit{Pozn:}
Průměrný počet pokusů je zaokrouhlen na tři desetinná místa.}
\end{tabular}
\caption{Výsledky algoritmů s větším prvním pokusem}\label{tabvysokyprvnipokus}
\end{table}


% \item aplikace problému mastermind jsou diskutovány v článku doerr playing mastermind with many colours na straně 4





