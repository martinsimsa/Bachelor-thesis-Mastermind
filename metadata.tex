%%% Vyplňte prosím základní údaje o závěrečné práci (odstraňte \xxx{...}).
%%% Automaticky se pak vloží na všechna místa, kde jsou potřeba.

% Druh práce:
%	"bc" pro bakalářskou
%	"mgr" pro diplomovou
%	"phd" pro disertační
%	"rig" pro rigorozní
\def\ThesisType{bc}

% Název práce v jazyce práce (přesně podle zadání)
\def\ThesisTitle{Logik - algoritmy a strategie}

% Název práce v angličtině
\def\ThesisTitleEN{Mastermind - algorithms and strategies}

% Jméno autora (vy)
\def\ThesisAuthor{Martin Šimša}

% Rok odevzdání
\def\YearSubmitted{2025}

% Název katedry nebo ústavu, kde byla práce oficiálně zadána
% (dle Organizační struktury MFF UK:
% https://www.mff.cuni.cz/cs/fakulta/organizacni-struktura,
% případně plný název pracoviště mimo MFF)
\def\Department{Katedra algebry}
\def\DepartmentEN{Department of Algebra}

% Jedná se o katedru (department) nebo o ústav (institute)?
\def\DeptType{Katedra}
\def\DeptTypeEN{Department}

% Vedoucí práce: Jméno a příjmení s~tituly
\def\Supervisor{doc. Mgr. Pavel Růžička, Ph.D.}

% Pracoviště vedoucího (opět dle Organizační struktury MFF)
\def\SupervisorsDepartment{Katedra algebry}
\def\SupervisorsDepartmentEN{Department of Algebra}

% Studijní program (kromě rigorozních prací)
\def\StudyProgramme{Matematika pro informační technologie}

% Nepovinné poděkování (vedoucímu práce, konzultantovi, tomu, kdo
% vám nosil pizzu a vařil čaj apod.)
\def\Dedication{Děkuji doc. Mgr. Pavlu Růžičkovi, Ph.D. za trpělivost při vedení práce a konstruktivní diskuzi. Za poskytnutí článku od autorů Koyama a Lai \cite{koyama} děkuji dr.~B.P.~Barteldu Kooi a Pavle Šimšové. Svým blízkým děkuji za podporu v každé fázi práce.}

% Abstrakt (doporučený rozsah cca 80-200 slov; nejedná se o zadání práce)
\def\Abstract{%
Práce se zabývá popisem algoritmů řešících hru Mastermind pro libovolný počet pozic a barev. Stav hry je reprezentován dvěma způsoby, jako posloupnost pokusů s ohodnocením a jako množina zbývajících možností na tajný kód. Vazby mezi stavy jsou v obou případech popsány v grafu a mezi oběma grafy je nalezena spojitost. Pomocí těchto grafů je definován strom, který reprezentuje nějaký algoritmus pro hru Mastermind. Dále je sestrojen obecný algoritmus sloužící k~popisu konkrétních algoritmů. Ve třetí kapitole jsou popsány tři deterministické algoritmy, které hru řeší. Jde o algoritmy Min-max, Max entropy a Most parts. Nakonec jsou tyto metody implementovány a otestovány na hře Mastermind o~čtyřech pozicích a šesti barvách.}

% Anglická verze abstraktu
\def\AbstractEN{%
This thesis describes algorithms solving the game Mastermind for~arbitrary number of positions and colours. The state of the game is described in~two ways, as a sequence of codes with evaluations and as a set of codes which could be the secret code. Both sets of state representations are described in a graph and~we show the relation between these graphs. Using that, a tree that represents a~particular algorithm for the game Mastermind is defined. Next, we construct a~general algorithm solving the game which we use for describing particular methods. In~the third chapter, three deterministic algorithms are described. They are called Min-max, Max-entropy and Most parts. Finally, we implement these algorithms and show their results in Mastermind with four positions and six colours.}

% 3 až 5 klíčových slov (doporučeno) oddělených \sep
% Hodí se pro nalezení práce podle tématu.
\def\ThesisKeywords{Logik\sep deterministický algoritmus\sep min-max\sep entropie}

\def\ThesisKeywordsEN{Mastermind\sep deterministic algorithm\sep min-max\sep entropy}

% Pokud některá z položek metadat obsahuje TeXové řídící sekvence, je potřeba
% dodat i verzi v obyčejném textu, která se objeví v metadatech formátu XMP
% zabudovaných do výstupního souboru PDF. Pokud si nejste jistí, prohlédněte si
% vygenerovaný soubor thesis.xmpdata.
\def\ThesisAuthorXMP{\ThesisAuthor}
\def\ThesisTitleXMP{\ThesisTitle}
\def\ThesisKeywordsXMP{\ThesisKeywords}
\def\AbstractXMP{\Abstract}

% Máte-li dlouhý abstrakt a nechceme se mu vejít na informační stranu,
% můžete použít toto nastavení ke zmenšení písma informační strany.
% (Uvažte nicméně zkrácení abstraktu, to je často lepší.)
\def\InfoPageFont{}
%\def\InfoPageFont{\small}  % odkomentujte pro zmenšení písma
