\chapter*{Úvod}
\addcontentsline{toc}{chapter}{Úvod}

Logik, anglicky Mastermind, je hra, která byla roku $1970$ vynalezena Mordecaiem Meirowitzem \cite{Nelson-history}. Je určena pro dva hráče. První hráč vytvoří čtyřmístný tajný kód složený z nějaké kombinace šesti barev. Druhý hráč se tento kód snaží nalézt na co nejmenší počet tahů. Po každém tahu dostane ohodnocení, jak se tah podobá tajnému kódu. Mastermind připomíná starší hru s názvem Bulls and Cows, která je založena na stejném principu. V tomto případě ale oba hráči vymyslí tajný kód a následně se střídají v hádání protivníkova kódu. 

Mastermind se dočkal veliké popularity a prodaly se desítky milionů kopií (T. Nelson, \cite{Nelson-history}). V návaznosti na úspěch vznikaly další varianty s různým počtem pozic a barev. Hra také přitáhla pozornost mnoha matematiků, kteří zkoumali různé algoritmy řešící základní hru o čtyř pozicích a šesti barvách řeší. Jedním z prvních byl Donald Knuth, který v roce $1977$ navrhl algoritmus zaručující nalezení tajného kódu na pět tahů\cite{donald_e__knuth_1977}. Déle trvalo nalézt optimální strategii z hlediska průměrného počtu pokusů. Tu nakonec objevili v roce $1993$ Kenji Koyama a Tony W. Lai. Ukázali, že nejlepší strategie dosahuje průměrného počtu pokusů $4.340$ na hru, a že tedy nelze zaručit dohrání hry na čtyři tahy \cite{koyama}. 

Přestože Mastermind byl vytvořen pro rekreační použití, metody sloužící k řešení hry lze aplikovat v reálném světě. Autoři R. Focardi a F. L. Luccio popsali spojitost mezi hrou Mastermind a prolamováním bankovních PINů \cite{Bank-pins-Focardi}. M. T. Goodrich poukázal na možná rizika při porovnávání sekvencí DNA pro hledání podobností mezi vzorky \cite{goodrich-dna}. 


Tato práce zobecňuje skupinu algoritmů, které hru Mastermind řeší. Kapitola~$1$ zavádí definice ohodnocení a průběhu hry. V kapitole $2$ je vytvořena terminologie sloužící k reprezentaci hry. Stavy hry jsou znázorněny jako vrcholy na grafu dvěma způsoby. Zaprvé jsou uchovávány jako posloupnosti zahraných pokusů s ohodnoceními. Druhá možnost je uchovávat množinu zbylých možností na tajný kód pro aktuální stav hry. Ukážeme spojitost mezi těmito dvěma způsoby pomocí operací a funkcí na grafech. Následně je zde vytvořen obecný algoritmus, pomocí kterého je popsán chod tří deterministických algoritmů popisovaných v kapitole $3$. Konkrétně jde o algoritmy navržené Donaldem Knuthem \cite{donald_e__knuth_1977}, Erichem Neuwirthem \cite{neuwirth} a Barteldem Kooi \cite{kooi}. V kapitole $4$ jsou uvedeny výsledky těchto algoritmů, které byly nalezeny pomocí programu vytvořeného pro tuto práci. Zkoumáme maximální a průměrné počty pokusů a také volbu prvního tahu. Diskutujeme tam také o případných variantách algoritmů. 